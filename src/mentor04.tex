\documentclass[11pt]{exam}
\usepackage{../commonheader}

\discnumber{4}
\title{Inheritance}
\date{September 18, 2017}

\begin{document}
\maketitle

\section{Dogs Yay}
\begin{questions}
\subimport{../topics/polymorphism/easy/}{dogs.tex}
\subimport{../topics/polymorphism/medium/}{poodle.tex}
\end{questions}

\clearpage

\section{An Appealing Appetizer}
\begin{questions}
\subimport{../topics/inheritance/easy/}{food.tex}
\end{questions}

\clearpage

\section{Iterator Interface}
\subimport{../topics/iterators/}{interface.tex}
\begin{questions}
\subimport{../topics/iterators/easy/}{array-iterator.tex}
\clearpage
\subimport{../topics/iterators/easy/}{intlist-iterator.tex}
\subimport{../topics/iterators/easy/}{print-all.tex}
\begin{solution}[0.7in]
\begin{blocksection}
\textbf{Meta:} In general, for all parts of this question, give students time
to work on the problem as well as prompting them to think about what they
\textbf{need} in order to start them off. 
 (i.e: What do you need to do to implement the Iterator Interface?
  What do you need to know for hasNext to work?)
\end{blocksection}
\end{solution}
\end{questions}

\clearpage

\section{\extra{Pokemon}}
\begin{questions}
\marginpar{\subimport{../topics/polymorphism/medium/pokemon/}{commands.tex}}
\subimport{../topics/polymorphism/medium/pokemon/}{question.tex}
\subimport{../topics/polymorphism/medium/pokemon/}{code.tex}
\end{questions}

\end{document}
