\begin{blocksection}
\question Let's model a feudal society where managers, merchants, and workers
cooperate to deliver goods. What happens when we call
\lstinline$new Manager().work()$?

\begin{lstlisting}[basicstyle=\ttfamily\footnotesize,
                   numberstyle=\ttfamily\tiny, numbers=left]
class Manager {
    Merchant merchant = new Merchant();
    Worker worker = new Worker();
    String[] goods = new String[1];
    void work() {
        try {
            merchant.trade(goods);
        } catch (RuntimeException e) {
            worker.produce("apple pie", goods);
            merchant.trade(goods);
            worker.produce("cornbread", goods);
        } catch (Exception e) {
            merchant.trade(goods);
        } finally {
            System.out.println("All in a day's work");
        }
    }
}
class Merchant {
    void trade(String[] goods) {
        try {
            for (String good : goods) {
                if (good != null) {
                    System.out.println("Traded 1 " + good);
                    return;
                }
            }
            throw new RuntimeException("Not enough goods");
        } catch (Exception e) {
            System.out.println("Oops");
            throw e;
        } finally {
            System.out.println("I love trading");
        }
    }
}
class Worker {
    void produce(String item, String[] goods) {
        int i = 0;
        while (goods[i] != null) {
            i += 1;
        }
        goods[i] = item;
        System.out.println("Done making 1 " + item);
    }
}
\end{lstlisting}
\end{blocksection}

\begin{solution}
\begin{verbatim}
Oops
I love trading
Done making 1 apple pie
Traded 1 apple pie
I love trading
All in a day's work
ArrayIndexOutOfBoundsException
\end{verbatim}

\textbf{Meta:} This is a walkthrough question. Giving students time to work
on their own usually results in one small mistake leading to a tangent off into
different aspects of the code. Highly recommended to have students chime in
as you step through the problem on the board, or through slides.

\textbf{Slides:}
\href{https://docs.google.com/presentation/d/1j418bduZS2Ltm6dVVg-b3WpGbOGrRUm6DKRkZCByuaQ/edit?usp=sharing}
     {Mudit's Slides}
\end{solution}
