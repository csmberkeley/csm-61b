\question
\begin{lstlisting}
try {
    doSomething();
} catch (ArrayIndexOutOfBoundsException e) {
    System.out.println("caught array index exception");
} catch (Exception e) {
    System.out.println("caught an exception");
    throw e;
} catch (NullPointerException e) {
    System.out.println("caught null pointer exception");
} finally {
    System.out.println("in finally block");
}
\end{lstlisting}

\begin{parts}
\part What will print if \texttt{doSomething()} throws a \texttt{NullPointerException}?
\begin{solution}[0.75in]
\begin{verbatim}
caught an exception
in finally block
NullPointerException
\end{verbatim}
\end{solution}

\part What if \texttt{doSomething()} throws an \texttt{ArrayIndexOutOfBoundsException}?
\begin{solution}[0.75in]
\begin{verbatim}
caught array index exception
in finally block
\end{verbatim}
\end{solution}

\part What if \texttt{doSomething()} doesn't error?
\begin{solution}[0.75in]
\begin{verbatim}
in finally block
\end{verbatim}
\end{solution}
\end{parts}
\begin{solution}
\textbf{Meta:} Remember that completing a catch prevents all other catches
in the same block from running. Even on a re-throw, no other catches in the
same block are called, because a catch already "handled" the exception. \\
\textbf{Slides:} \href{https://docs.google.com/presentation/d/1j418bduZS2Ltm6dVVg-b3WpGbOGrRUm6DKRkZCByuaQ/edit?usp=sharing}{Mudit's Slide Doc}
\end{solution}
