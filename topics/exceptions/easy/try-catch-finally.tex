\begin{blocksection}
\question

\begin{lstlisting}
try {
    doSomething();
} catch (ArrayIndexOutOfBoundsException e) {
    System.out.println("caught array index exception");
} catch (Exception e) {
    System.out.println("caught an exception");
    throw e;
} catch (NullPointerException e) {
    System.out.println("caught null pointer exception");
} finally {
    System.out.println("in finally block");
}
\end{lstlisting}

\begin{solution}
Note that the code is a trick. Java doesn't like ambiguity or unreachable code,
so the \lstinline$catch (NullPointerException e)$ causes a compile-time error
as \lstinline$NullPointerException$ is a subclass of \lstinline$Exception$.

The rest of the questions run assuming that the catch block for the
\lstinline$NullPointerException$ is not present.
\end{solution}

\begin{parts}
\part What will print if \lstinline$doSomething()$ throws a
\lstinline$NullPointerException$?
\begin{solution}[0.75in]
\begin{verbatim}
caught an exception
in finally block
NullPointerException
\end{verbatim}
\end{solution}

\part What if \lstinline$doSomething()$ throws an
\lstinline$ArrayIndexOutOfBoundsException$?
\begin{solution}[0.75in]
\begin{verbatim}
caught array index exception
in finally block
\end{verbatim}
\end{solution}

\part What if \lstinline$doSomething()$ doesn't error?
\begin{solution}[0.75in]
\begin{verbatim}
in finally block
\end{verbatim}
\end{solution}
\end{parts}

\begin{solution}
\textbf{Meta:} Remember that completing a catch prevents all other catches
in the same block from running. Even on a re-throw, no other catches in the
same block are called, because the exception has already been `handled'!

\textbf{Slides:}
\href{https://docs.google.com/presentation/d/1j418bduZS2Ltm6dVVg-b3WpGbOGrRUm6DKRkZCByuaQ/edit?usp=sharing}
     {Mudit's Slides}
\end{solution}
\end{blocksection}
