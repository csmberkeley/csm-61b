\begin{blocksection}
\question Suppose you've been asked to test the following code snippet. Recall
that a \define{leap year} is divisible by 4 except for 3 out of every 4
\emph{centuries} where the year is not divisible by 400.

\begin{lstlisting}
if (year % 400 != 0) {
    System.out.println(year + " is not a leap year");
} else if (year % 100 != 0) {
    System.out.println(year + " is a leap year");
} else if (year % 4 != 0) {
    System.out.println(year + " is not a leap year");
} else {
    System.out.println(year + " is a leap year");
}
\end{lstlisting}

\begin{parts}
\part Provide a method header (or, in Python, the \emph{function definition})
for the code above.
\begin{solution}[0.5in]
\lstinline$public static void printLeapYear(int year)$
\end{solution}

\part Write a \lstinline$main$ method that tests the correctness of the code
against a couple inputs that work as expected and a couple inputs that cause
problems.
\begin{solution}[2in]
\begin{lstlisting}
public static void main(String[] args) {
    printLeapYear(1900);
    printLeapYear(1904);
    printLeapYear(1905);
    printLeapYear(1908);
}
\end{lstlisting}
\end{solution}

\part \define{Refactor} the program such that it functions correctly and can be
more easily tested in the future.
\begin{solution}[3in]
\begin{lstlisting}
public static void printLeapYear(int year) {
    if (isLeapYear(year)) {
        System.out.println(year + " is a leap year");
    } else {
        System.out.println(year + " is not a leap year");
    }
}
private static boolean isLeapYear(int year) {
    if (year % 4 != 0) {
        return false;
    } else if (year % 100 != 0) {
        return true;
    } else if (year % 400 != 0) {
        return false;
    } else {
        return true;
    }
}
\end{lstlisting}
\end{solution}
\end{parts}
\end{blocksection}
