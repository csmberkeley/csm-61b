\question How is this implementation of \texttt{swap} different?

\begin{lstlisting}
class Coordinate {
    int x, y;
    Coordinate(int x, int y) {
        this.x = x;
        this.y = y;
    }
}
\end{lstlisting}

\begin{lstlisting}
class SwapObject {
    public static void swap(Coordinate p) {
        int temp = p.x;
        p.x = p.y;
        p.y = temp;
    }
    public static void main(String[] args) {
        Coordinate p = new Coordinate(2, 5);
        System.out.println("p.x: " + p.x + ", p.y: " + p.y);
        swap(p);
        System.out.println("p.x: " + p.x + ", p.y: " + p.y);
    }
}
\end{lstlisting}

\begin{solution}
When calling \lstinline$swap$ with a \lstinline$Coordinate$ object, we're passing a reference to the original \lstinline$Coordinate$ object. The object's instance variables can be changed from within \lstinline$swap$ and will remain changed after we exit from the function.

\textbf{Meta:} It can be useful to mention that Java stores only the address of an object in a variable of non-primitive type, and so when passed to a function, the address of the original object gets passed around. 
\end{solution}
