\question In general, there are 4 ways to heapify. Which 2 ways actually work?

\begin{itemize}
\item Level order, bubbling up
\item Level order, bubbling down
\item Reverse level order, bubbling up
\item Reverse level order, bubbling down
\end{itemize}

\begin{solution}
Only level order, bubbling up and reverse level order, bubbling down work as
they maintain heap invariant. Namely, that every node is either larger (in a
max heap) or smaller (in a min heap) than all of its children.

Meta: Students often ask about the runtime of these methods. The specific runtime for heapification is hard to prove, but specifically level order, bubbling up will take $O(N\log(N))$ and reverse level order, bubbling down takes $O(N)$, so reverse level order + bubbling down is faster. Intuition on why it is faster: the majority of the nodes are near the bottom of the tree, so a lot of the nodes bubble down really quickly. 
\end{solution}
