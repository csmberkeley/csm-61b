\begin{blocksection}
\question

\begin{lstlisting}
// Assume M and N are both large.
public static int crimsonTuna(int[][] array) {
    if (array.length < 4) {
        return 0;
    }
    for (int i = 0; i < array.length; i++) {
        for (int j = 0; j < array[i].length; j++) {
              if (i == 4) {
                return -1;
            }
        }
    }
    return 1;
}
\end{lstlisting}
\end{blocksection}

\begin{blocksection}
\begin{solution}
$\Theta(N)$. This function returns after it reaches the fourth row (and if there are
less than four rows it returns immediately). The number of elements it's able
to reach in four rows is dependent on the number of columns, which in this case
is $N$. \newline

Note: If we did not specify that M and N were both large, we would need to give an
$O$ bound instead of a $\Theta$ bound here. The reason is that we would need to
consider all three cases where the size of the input gets large: \newline 

1. $M$ approaches infinity, $N$ is small \newline
2. $N$ approaches infinity, $M$ is small \newline
3. $M$ and $N$ both approach infinity \newline

If all three cases have the same $\Omega$ and $O$ bound then we can write a
$\Theta$ bound for this problem. However, if we consider case 2, we can see
that if $M < 4$, then the function will return immediately after the first
if statement. Thus not all three cases have the same  $\Omega$ and $O$ bound,
so we can only provide an $O$ bound for the runtime.
\end{solution}
\end{blocksection}
