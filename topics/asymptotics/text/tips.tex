\begin{enumerate}
\item Before attempting to calculate a function's runtime, first try to understand what the function does.
\item Try some small sample inputs to get a better intuition of what the function's runtime is like. What is the function doing with the input? How does the runtime change as the input size increases? Can you spot any 'gotchas' in the code that might invalidate our findings for larger inputs?
\item Try to lower bound and upper bound the function runtime given what you know about how the function works. This is a good sanity check for your later observations.
\item If the function is recursive, draw a call tree to map out the recursive calls. This breaks the problem down into smaller parts that we can analyze individually. Once each part of the tree has been analyzed, we can then reassemble all the parts to determine the overall runtime of the function.
\item If the function has a complicated loop, draw a bar graph to map out how much work the body of the loop executes for each iteration.
\end{enumerate}
