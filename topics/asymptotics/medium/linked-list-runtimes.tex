\marginpar{\subimport{../../linked-lists/naive/}{code.tex}
           \subimport{../../linked-lists/encapsulated/}{singly-linked.tex}
           \subimport{../../linked-lists/encapsulated/}{doubly-linked.tex}}
\begin{blocksection}
\question

Given a linked list of length $N$, give a tight asymptotic runtime
bound for each operation. Recall that \lstinline$IntList$ is a naive linked
list, \lstinline$SLList$ is an encapsulated singly-linked list with a front
sentinel, and \lstinline$DLList$ is an encapsulated doubly-linked list with
front and back pointers.
\end{blocksection}
\ifprintanswers
{
\renewcommand{\arraystretch}{2}
\setlength{\tabcolsep}{16pt}
\begin{tabularx}{\textwidth}{Xccc}
Operation                          & \lstinline$IntList$    & \lstinline$SLList$     & \lstinline$DLList$     \\ \hline
\lstinline$size()$                 & $\color{red}\Theta(N)$ & $\color{red}\Theta(1)$ & $\color{red}\Theta(1)$ \\
\lstinline$get(int index)$         & $\color{red}O(N)$      & $\color{red}O(N)$      & $\color{red}O(N)$      \\
\lstinline$addFirst(E e)$          & $\color{red}\Theta(1)$ & $\color{red}\Theta(1)$ & $\color{red}\Theta(1)$ \\
\lstinline$addLast(E e)$           & $\color{red}\Theta(N)$ & $\color{red}\Theta(N)$ & $\color{red}\Theta(1)$ \\
\lstinline$addBefore(E e, Node n)$ & $\color{red}O(N)$      & $\color{red}O(N)$      & $\color{red}\Theta(1)$ \\
\lstinline$remove(int index)$      & $\color{red}O(N)$      & $\color{red}O(N)$      & $\color{red}O(N)$      \\
\lstinline$remove(Node n)$         & $\color{red}O(N)$      & $\color{red}O(N)$      & $\color{red}\Theta(1)$ \\
\end{tabularx}

\color{red} size(): SLList and DLList both save size as a field, whereas IntList has to iterate through the whole list to find the size. \\\\
addBefore(E e, Node n): DLList has a pointer from each node to the next node and to the previous node, so adding before is constant time. IntList and SLList  have to iterate through the list from the start to find the node before the current node, in order to add before. \\\\
remove(Node n): Because DLList has a pointer to the previous node and next node, it can easily remove the node and have it's previous node point to its next node (and have its next node point to its previous node). However, IntList and SLList have to iterate through the list to find the node before the current one.\\

}
\else
{
\renewcommand{\arraystretch}{2}
\setlength{\tabcolsep}{16pt}
\begin{tabularx}{\textwidth}{Xccc}
Operation                          & \lstinline$IntList$ & \lstinline$SLList$ & \lstinline$DLList$ \\ \hline
\lstinline$size()$                 &                     &                    &                    \\
\lstinline$get(int index)$         &                     &                    &                    \\
\lstinline$addFirst(E e)$          &                     &                    &                    \\
\lstinline$addLast(E e)$           &                     &                    &                    \\
\lstinline$addBefore(E e, Node n)$ &                     &                    &                    \\
\lstinline$remove(int index)$      &                     &                    &                    \\
\lstinline$remove(Node n)$         &                     &                    &                    \\
\end{tabularx}
}
\fi


\begin{parts}
\part Give the runtime of \lstinline$addAll(Collection<E> c)$ assuming an empty
linked list and \lstinline$c$ of size $N$. Assume \lstinline$addAll$ just calls
\lstinline$addLast$ repeatedly.

\begin{solution}[1in]
\lstinline$IntList$: $\Theta(N^2)$ \\
\lstinline$SLList$: $\Theta(N^2)$

IntList and SLList both take $\Theta(N)$ time to addLast(E e), where N is number of items currently in the list.
The trick here is to notice that we start adding to an empty list and add until we fill the list to size N. 
So every addition takes variable time since the size of the list is different at each addition. 
We can calculate the size of the list before each addition as follows:

$0 + 1 + 2 + ... + N-1 = \frac{N(N-1)}{2}$

Therefore, it takes $\Theta(N^2)$ time. 

\lstinline$DLList$: $\Theta(N)$
\end{solution}

\part How can we do better?

\begin{solution}[1in]
Instead of calling \lstinline$addLast$, keep track of the last node and append
each new element in constant time to the end of the list.
\end{solution}
\end{parts}
