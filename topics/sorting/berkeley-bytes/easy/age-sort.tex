\question Now, for taxes, you need to submit a list of the ages of your customers in sorted order. Define a procedure, \texttt{ageSort}, which takes an \texttt{int[] array} of all customers' ages and returns a sorted array. Assume customers are less than 150 years old.

\ifprintanswers
\else
\begin{lstlisting}
public class BerkeleyBytes {
    public static int[] histogram(int[] ages) {
        int[] ageCounts = new int[150];
        for (int age : ages) {
            ageCounts[age - 1] += 1;
        }
        return ageCounts;
    }
    public static int[] ageSort(int[] ages) {












    }
}
\end{lstlisting}
\fi

\begin{solution}
\begin{lstlisting}
public class BerkeleyBytes {
    private static int maxAge = 150;
    public static int[] histogram(int[] ages) {
        int[] ageCounts = new int[maxAge];
        for (int age : ages) {
            ageCounts[age - 1] += 1;
        }
        return ageCounts;
    }
    public static int[] ageSort(int[] ages) {
        int[] ageCounts = histogram(ages);
        int[] result = new int[ages.length];
        int index = 0;
        for (int age = 0; age < maxAge; age++) {
            for (int count = 0; count < ageCounts[age]; count++) {
                result[index] = age + 1;
                index += 1;
            }
        }
        return result;
    }
}
\end{lstlisting}
\end{solution}