\question Design a \texttt{hashCode} for binary strings which contain only the characters \texttt{0} or \texttt{1}.

\ifprintanswers
\else
\begin{lstlisting}
public class BinString {
    private String value;
    public int hashCode() {









\end{lstlisting}
\fi

\begin{solution}
\begin{lstlisting}
public class BinString {
    private String value;
    public int hashCode() {
        int hash = 0;
        for (char c : value.toCharArray()) {
            hash = 2 * hash + (c - '0');
        }
        return hash;
    }
}
\end{lstlisting}

This \texttt{hashCode} implementation returns the decimal value represented by the binary string.
For example, the hash code of \texttt{010} is 2 and the hash code of \texttt{11111} is 31.

Caveat: This \texttt{hashCode} is not perfect.
Leading zeros aren't taken into account: \texttt{1}, \texttt{001}, and \texttt{0000000001} all have a \texttt{hashCode} of 3.
One possible solution is to simply prepend a \texttt{1} to each string.
\end{solution}