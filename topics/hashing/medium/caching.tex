\question You are a software engineer for a newspaper company! Your users are complaining about how slowly your website loads. After performing some performance profiling, you realize that the database queries are slowing the system down.

To fix the issue, you decide to implement a cache that contains the most recently accessed articles. The cache is only fast if it's small so you can only store a maximum of $N$ articles. You want to keep only the $N$ most recent articles that people have read. If a new, unique article is accessed, then the oldest article should be replaced.

Describe how you would implement this cache. What combinations of data structures would you use to build this efficiently? 

\begin{solution}[2in]
Use a \texttt{HashMap} with references to nodes in a doubly linked list. The \texttt{HashMap} will map the name of the article to a tuple containing the article as well as a reference to a node in a doubly linked list. If the article we are trying to access is in the \texttt{HashMap}, we deliver the article, and then retrieve its respective node in the linked list and move it to the front. If the article is not in the \texttt{HashMap}, and the size of the \texttt{HashMap} is still less than $N$, we can fetch from the main server and create a new node to append to the front of the linked list. If the \texttt{HashMap} is at capacity, we look at the tail of the linked list and delete that article from both the linked list as well as the \texttt{HashMap}. We then add the new article into the \texttt{HashMap} and to the front of the linked list. In Java there's actually a data structure for this: \texttt{java.util.LinkedHashMap}!
\end{solution}