
\question Which of the following \lstinline$hashCodes$ are valid?

\begin{lstlisting}
class Point {
    private int x, y;
    private static int count = 0;
    public Point(int x, int y) {
        this.x = x;
        this.y = y;
        count += 1;
    }
}
\end{lstlisting}

\begin{parts}
\part What does it mean for a hashcode to be valid?
\begin{solution}[0.75in]
The general contract of hashCode is:
\begin{itemize}
\item Whenever it is invoked on the same object more than once, the hashCode method must consistently returns the same integer if nothing about the object changes.
\item If two objects are equal according to the equals(Object) method, then calling the hashCode method on each of the two objects must produce the same integer result.
\item It is not required that if two objects are unequal according to the equals(java.lang.Object) method, then calling the hashCode method on each of the two objects must produce distinct integer results. However, the programmer should be aware that producing distinct integer results for unequal objects may improve the performance of hash tables.
\end{itemize}
\end{solution}

\part
\begin{lstlisting}
public void hashCode() {
    System.out.print(this.x + this.y);
}
\end{lstlisting}
\begin{solution}[0.25in]
Invalid. Return type should be \lstinline$int$.
\end{solution}

\part
\begin{lstlisting}
public int hashCode() {
    Random random = new Random();
    return random.nextInt();
}
\end{lstlisting}
\begin{solution}[0.25in]
Invalid. Not deterministic.
\end{solution}

\part
\begin{lstlisting}
public int hashCode() {
    return this.x + this.y;
}
\end{lstlisting}
\begin{solution}[0.25in]
Valid, but certain inputs may cause a significant number of collisions.
\end{solution}

\part
\begin{lstlisting}
public int hashCode() {
    return count;
}
\end{lstlisting}
\begin{solution}[0.25in]
Invalid. Not consistent.
\end{solution}

\part
\begin{lstlisting}
public int hashCode() {
    return 4;
}
\end{lstlisting}
\begin{solution}[0.25in]
Valid, but causes collisions on any input.
\end{solution}
\end{parts}

