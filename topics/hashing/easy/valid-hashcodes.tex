\begin{blocksection}
\question What does it mean for a hashcode to be valid?
\begin{solution}[0.75in]
\begin{itemize}
\item Consistency: Whenever it is invoked on the same object more than once,
the \lstinline$hashCode$ method must consistently returns the same integer if
nothing about the object changes.
\item Equality constraint: If two objects are equal according to the
\lstinline$equals(Object)$ method, then calling the \lstinline$hashCode$ method
on each of the two objects must produce the same integer result.
\item It is not required that if two objects are unequal according to the
\lstinline$equals(Object)$ method, then calling the \lstinline$hashCode$ method
on each of the two objects must produce distinct integer results. However, the
programmer should be aware that producing distinct integer results for unequal
objects may improve the performance of hash tables.
\end{itemize}
\end{solution}
\end{blocksection}

\question Which of the following hashcodes are valid? Good?

\begin{lstlisting}
class Point {
    private int x, y;
    private static int count = 0;
    public Point(int x, int y) {
        this.x = x;
        this.y = y;
        count += 1;
    }
}
\end{lstlisting}

\begin{parts}
\part
\begin{lstlisting}
public void hashCode() {
    System.out.print(this.x + this.y);
}
\end{lstlisting}
\begin{solution}[0.25in]
Invalid. Return type should be \lstinline$int$. The method should override
Java's \lstinline$hashCode$ method, but since it has the same signature with
different return types, this would be a compile-time error
\end{solution}

\part
\begin{lstlisting}
public int hashCode() {
    Random random = new Random();
    return random.nextInt();
}
\end{lstlisting}
\begin{solution}[0.25in]
Invalid. Not consistent. This method would compile because it does return an
\lstinline$int$. However, this does not obey the principle of consistency about
returning the same integer every time, and so the \lstinline$HashMap$ would not
work as expected. If we store an object at key 1, the next time around the
object might have key 2, and we would be unable to find the object.
\end{solution}

\part
\begin{lstlisting}
public int hashCode() {
    return this.x + this.y;
}
\end{lstlisting}
\begin{solution}[0.25in]
Valid, but certain inputs may cause a significant number of collisions. This
method would work, since it obeys all bullets under the contract. However, this
will cause many collisions. All points that sum up to the same number will be
mapped to the same \lstinline$hashCode$ such as $(5, 3), (4, 4), (2, 6),
(6, 2), (1, 7), (7, 1), \ldots$
\end{solution}

\part
\begin{lstlisting}
public int hashCode() {
    return count;
}
\end{lstlisting}
\begin{solution}[0.25in]
Invalid. Not consistent. This method would not error, because it does return an
\lstinline$int$, but it does not obey the principle of consistency.
\lstinline$count$ is a static variable. Consider what happens when we create a
\lstinline$new Point(3, 4)$ and call the \lstinline$hashCode$ method on it the
first time. We would get 1. If we then create a \lstinline$new Point(1, 2)$,
calling the \lstinline$hashCode$ method on the earlier \lstinline$Point(3, 4)$
would now return 2, even though nothing about the original \lstinline$Point$
object changed.
\end{solution}

\part
\begin{lstlisting}
public int hashCode() {
    return 4;
}
\end{lstlisting}
\begin{solution}[0.25in]
Valid, but causes collisions on all inputs. While this method is valid, it
collides on any input.
\end{solution}
\end{parts}
