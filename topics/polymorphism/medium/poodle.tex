\begin{blocksection}
\question Consider the following class definitions. What would each call in \lstinline$Poodle.main$ print? If a line would cause an error, determine if it is a compile-error or runtime-error.

\begin{lstlisting}
class Dog {
    void bark(Dog dog) {
        System.out.println("bark");
    }
}

class Poodle extends Dog {
    void bark(Dog dog) {
        System.out.println("woof");
    }
    void bark(Poodle poodle) {
        System.out.println("yap");
    }
    void play(Dog dog) {
        System.out.println("no");
    }
    void play(Poodle poodle) {
        System.out.println("bowwow");
    }
    public static void main(String[] args) {
        Dog dog = new Poodle();
        Poodle poodle = new Poodle();
\end{lstlisting}

\ifprintanswers\else
\begin{lstlisting}
        dog.play(dog)
        dog.play(poodle)
        poodle.play(dog)
        poodle.play(poodle)

        poodle.bark(dog)
        poodle.bark(poodle)
        dog.bark(dog)
        dog.bark(poodle)
\end{lstlisting}
\fi

\begin{solution}
\begin{lstlisting}
        dog.play(dog)           Compile-error
        dog.play(poodle)        Compile-error
        poodle.play(dog)        no
        poodle.play(poodle)     bowwow

        poodle.bark(dog)        woof
        poodle.bark(poodle)     yap
        dog.bark(dog)           woof
        dog.bark(poodle)        woof
\end{lstlisting}
\end{solution}

\begin{lstlisting}
    }
}
\end{lstlisting}
\begin{solution}
\textbf{Meta:} This question will require a walkthrough \\
First set up the initial rules: dynamic method lookup, method overriding,
static and dynamic types, and compile-time vs run-time errors.
Give students a chance to work on their own, then clear up misconceptions
\end{solution}
\end{blocksection}
