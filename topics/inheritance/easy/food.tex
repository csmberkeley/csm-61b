\question
\begin{lstlisting}
public interface Consumable {
    public void consume();
}
public abstract class Food implements Consumable {
    String name;
    public abstract void prepare();
    public void play() {
        System.out.println("Mom says, 'Don't play with your food.'");
    }
}
public class Snack extends Food {
    public void prepare() {
        System.out.println("Taking " + name + " out of wrapper");
    }
    public void consume() {
        System.out.println("Snacking on " + name);
    }
}
\end{lstlisting}

\begin{parts}
\begin{blocksection}
\part Compare and contrast interfaces and abstract classes.

\begin{solution}[1in]
\begin{itemize}
\item Java classes cannot extend multiple superclasses (unlike Python) but
classes can implement multiple interfaces.
\item Interfaces are implicitly public.
\item Interfaces can't have fields declared as instance variables; any fields
that are declared are implicitly \lstinline$static$ and \lstinline$final$.
\item Interfaces use the \lstinline$default$ keyword to declare \emph{concrete}
implementations while abstract classes use the \lstinline$abstract$ keyword to
declare \emph{abstract} implementations.
\item Interfaces define the way we interact with an implementing object or
functions of an object. Conversely, abstract classes define an ``is-a''
relationship and tell us more about the object's fundamental identity.
\end{itemize}

\textbf{Meta:} Warn students to be careful searching about this topic as CS 61B
uses Java 8 but most content online covers Java 7, which behaves differently.
\end{solution}
\end{blocksection}

\begin{blocksection}
\part Do we need the \lstinline$play$ method in \lstinline$Snack$?

\begin{solution}[0.5in]
No, we do not need the \lstinline$play$ method because it's already defined in
the abstract class. Java will lookup the parent class's method if it cannot
find it in the child class.
\end{solution}
\end{blocksection}

\begin{blocksection}
\part Does this compile? \lstinline$Consumable chips = new Snack();$

\begin{solution}[0.5in]
Yes, the code compiles since \lstinline$Snack$ inherits from the
\lstinline$Food$ class which implements the \lstinline$Consumable$ interface.
\end{solution}
\end{blocksection}
\end{parts}

\begin{solution}
\textbf{Meta:} Use this question as a mini-lecture for interfaces to build up
to the next question. This question should not take a lot of time.
\end{solution}
