\question What is the pre-order, in-order, and post-order traversal of a tree?
\newline
\newline
\newline
\question Given a node in a binary search tree (with parent pointers), write a \texttt{successor} method which returns the next node in the in-order traversal of the BST. If there is no successor, simply return null.

\ifprintanswers
\else
\begin{lstlisting}
public class BinarySearchTree<T extends Comparable<T>> {
    protected Node root;
    protected class Node {
        public T value;
        public Node parent;
        public Node left;
        public Node right;
    }
    private Node successor(Node node) {


















\end{lstlisting}
\fi

\begin{solution}
\begin{lstlisting}
public class BinarySearchTree<T extends Comparable<T>> {
    protected Node root;
    protected class Node {
        public T value;
        public Node parent;
        public Node left;
        public Node right;
    }
    
The successor of a node will always be either the leftmost branch of the node's right branch, or the node's first ancestor that does not have a parent or does not have a right branch.
    
    private Node successor(Node node) {
        if (node.right != null) {
            node = node.right; // node's right branch
            while (node.left != null) {
                node = node.left; // finding leftmost branch of node's right branch
            }
            return node;
        } else {
            Node parent = node.parent;
            while (parent != null && parent.right == node) { 
            // while loop stops when the node doesn't have a parent or a right branch
                node = parent;
                parent = parent.parent; 
            }
            return parent;
        }
    }
}
\end{lstlisting}
\end{solution}