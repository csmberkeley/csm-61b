\begin{blocksection}
\question Given a node in a binary search tree (with parent pointers),
implement \lstinline$successor$ which returns the next node in the in-order
traversal of the BST. If there is no successor, return null.

\ifprintanswers
\else
\begin{lstlisting}
public class BinarySearchTree<T extends Comparable<T>> {
    protected Node root;
    protected class Node {
        public T value;
        public Node parent;
        public Node left;
        public Node right;
    }
    private Node successor(Node node) {


















\end{lstlisting}
\fi

\begin{solution}[2.5in]
\begin{lstlisting}
public class BinarySearchTree<T extends Comparable<T>> {
    protected Node root;
    protected class Node {
        public T value;
        public Node parent;
        public Node left;
        public Node right;
    }
    private Node successor(Node node) {
        if (node.right != null) {
            node = node.right;          // node's right branch
            while (node.left != null) {
                node = node.left;       // finding leftmost branch of node's right branch
            }
            return node;
        } else {
            Node parent = node.parent;
            while (parent != null && parent.right == node) {
                // stop when the node doesn't have a parent or a right branch
                node = parent;
                parent = parent.parent;
            }
            return parent;
        }
    }
}
\end{lstlisting}

There are two cases to find the next node in the in-order traversal. Let's think about the case specifically for Binary Search Trees. The next node in the in-order traversal is the same as the next biggest element in the Binary Search Tree.

The first case is when the node passed in has a right branch. If it has a right branch, then the next largest element must lie in the sub-tree whose root is the right-child of the given node. Let's call this sub-tree T. The final thing to realize is that the element we are looking for is the smallest element in this sub-tree T. This is why we traverse left-wards, until we reach the left-bottom of T, which is the node we need to return.

The second case is when the node passed in has no right branch. To reason about this case, first consider a simple case with just 2 nodes. 

\begin{center}
\begin{tikzpicture}[very thick,level/.style={sibling distance=20mm/#1},level distance=30pt]
\node[vertex]{$3$}
child {
    node[vertex]{$2$}
    child {
        node[vertex]{$1$}
    }
}
\end{tikzpicture}
\end{center}

Say that the node passed in is Node 1. Since the node doesn't have a right child, the next-largest node must be somewhere above it. In this example, we see that the answer is Node 2. As we traverse up from Node 2 to Node 1, we realize that the edge between 2 and 1 is a left-edge (1 is a left-child of 2). The general argument here is: keep traversing up the parents. As soon as we traverse a left-edge, we can stop and return the parent node (in this case, Node 2). Let's traverse up the parents, and name the first left-edge we encounter $E$, the parent on this edge $P$, and the child $C$. Why can we be sure that $P$ is the node to return? 

The first thing to note is that as we traverse up the tree, if we don't use a left-edge, then the child is greater than the parent. This is not what we want, since we want the next largest node. The next thing to notice is that going up a left-edge means that the parent is necessarily greater, by the definition of a left-edge in a Binary Search Tree. All that's left to argue now is that if node $N$ is passed in, then $P>N$, and that there is no other node that is smaller than $P$, but greater than $N$. Consider the following example:

\begin{center}
\begin{tikzpicture}[very thick,level/.style={sibling distance=20mm/#1},level distance=30pt]
\node[vertex]{$4$}
child {
    node[vertex]{$1$}
    child {edge from parent[draw=none]}
    child {
        node[vertex]{$2$}
    }
} 
\end{tikzpicture}
\end{center}

If the node that we're passed in is Node 2, then the first left-edge we will traverse while going up would have 4 as the parent. This example helps us see that the left-edge's parent will be greater, i.e., $P>N$, since it is guaranteed that $N$ lies in the left-subtree of $P$. Note that we don't need to consider any of $P$'s ancestors. If $P$ has a parent through a right-edge, then that parent is bigger than $P$, which is not what we want. If $P$ has a parent through a left-edge, then that parent is smaller than $P$, but it is also smaller than $N$, since $N$ would lie in the left-subtree of $P$'s parent.  

The final argument is that there is no other node that is smaller than $P$, but greater than $N$. Let's look back at the example. First, we look at Node 1. Node 1 is smaller than Node 2, since it is a right-edge. In the example, Node 1 does not have any left-children, but if it did, we can be sure that they would all, too, be smaller than Node 2. So, any time we traverse a right-edge, we can ignore not only the right-edge's parent, but also be sure that everything in the left subtree of the right-edge can be safely ignored. This concludes the argument, since we have shown that right-edges or their subtrees do not contain any nodes that are smaller than $P$ but larger than $N$. 

\end{solution}
\end{blocksection}
