\begin{blocksection}
\question Given a node in a binary search tree (with parent pointers),
implement \lstinline$successor$ which returns the next node in the in-order
traversal of the BST. If there is no successor, return null.

\ifprintanswers
\else
\begin{lstlisting}
public class BinarySearchTree<T extends Comparable<T>> {
    protected Node root;
    protected class Node {
        public T value;
        public Node parent;
        public Node left;
        public Node right;
    }
    private Node successor(Node node) {


















\end{lstlisting}
\fi

\begin{solution}[2.5in]
\begin{lstlisting}
public class BinarySearchTree<T extends Comparable<T>> {
    protected Node root;
    protected class Node {
        public T value;
        public Node parent;
        public Node left;
        public Node right;
    }
    private Node successor(Node node) {
        if (node.right != null) {
            node = node.right;          // node's right branch
            while (node.left != null) {
                node = node.left;       // finding leftmost branch of node's right branch
            }
            return node;
        } else {
            Node parent = node.parent;
            while (parent != null && parent.right == node) {
                // stop when the node doesn't have a parent or a right branch
                node = parent;
                parent = parent.parent;
            }
            return parent;
        }
    }
}
\end{lstlisting}

The successor of a node will always be either the leftmost branch of the node's
right branch, or the node's first ancestor that either does not have a parent
or does not have a right branch.
\end{solution}
\end{blocksection}
