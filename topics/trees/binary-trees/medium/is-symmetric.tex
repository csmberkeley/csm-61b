\begin{blocksection}
\question Define \lstinline$isSymmetric$ which checks whether the binary tree
is a mirror of itself.

\ifprintanswers\else
\begin{lstlisting}
public boolean isSymmetric() {
    if (root == null) {
        return true;
    }
    return isSymmetric(root.left, root.right); // use helper method
}

private boolean isSymmetric(Node left, Node right) {












}
\end{lstlisting}
\fi

\begin{solution}
\begin{lstlisting}
public boolean isSymmetric() {
    if (root == null) {
        return true;
    }
    return isSymmetric(root.left, root.right); // use helper method
}

private boolean isSymmetric(Node left, Node right) {
    if (left == null) {
        return right == null; // if left is null, right must also be null
    } else if (right == null) {
        return false;         // left is not null but right is null, so not symmetric
    } else if (!left.value.equals(right.value)) {
        return false;         // left value and right value are unequal
    } else {
        return isSymmetric(left.right, right.left) &&
               isSymmetric(left.left, right.right);
    }
}
\end{lstlisting}

\textbf{Meta}: We can use a helper function here to create a new method that
takes in two parameters, the left and the right branch of the current tree we
are rooted at.  This allows us to more easily compare the content of the two
branches to see if they are the same.
\end{solution}
\end{blocksection}
