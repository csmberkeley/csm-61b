\question
A \textbf{queue} is a data structure that orders items in a first-in-first-out (FIFO) manner, meaning that the first element you \lstinline$add$ will be at the front and the last item you \lstinline$add$ will be at the tail.   

A \textbf{stack} is a data structure that orders items in a last-in-first-out (LIFO) manner, meaning that the first element you \lstinline$add$ will be at the tail and the last item you \lstinline$add$ will be at the front.

Calling $remove$ on either a queue or a stack will return the item ordered at the beginning.  For a queue, this would be the least recently inserted item, whereas for a stack, this would be the most recently inserted item.

\begin{lstlisting}
public class BinaryTree<T> {
    protected Node root;
    protected class Node {
        public T value;
        public Node left;
        public Node right;
    }
    public void treeTraversal(Fringe<Node> fringe) {
        fringe.add(root);
        while (!fringe.isEmpty()) {
            Node node = fringe.remove();
            System.out.println(node.value);
            if (node.left != null) {
                fringe.add(node.left);
            }
            if (node.right != null) {
                fringe.add(node.right);
            }
        }
    }
}
\end{lstlisting}

\marginpar{\subimport{../}{example-tree.tex}}

What will Java display?

\begin{parts}
\part \texttt{tree.traversal(new Queue<Node>());}
\begin{solution}[1in]
\begin{verbatim}
1
2
7
5
9
3
4
\end{verbatim}
Notice that this solution goes in breadth-first order.  We see that we start at the root $1$, and then proceed to traverse the tree level-by-level.  The second level is $2$ and $7$, which are both one hop away from the root.  The third level is $5$, $9$, and $3$, which are all two hops away from the root.  Lastly, we have $4$, which is three hops away from the root.
\end{solution}

\part \texttt{tree.traversal(new Stack<Node>());}
\begin{solution}[1in]
\begin{verbatim}
1
7
3
4
2
9
5
\end{verbatim}
Notice that this solution goes in depth-first order.  We first traverse all the way down the right branch: $1$, $7$, $3$, $4$.  Once we reach $4$, the leaf node, we go back up to the root and traverse all the way down the left branch: $2$, $9$, $5$.  We always traverse the right branch first due to the way the code is written (the right branch is added to the stack after the left branch, meaning the right branch would be explored first in LIFO ordering).
\end{solution}
\end{parts}